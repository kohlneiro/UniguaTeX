%**********************************************%
%			      REFERÊNCIAS			      	%                      
%**********************************************%
\usepackage[alf, abnt-emphasize=bf, bibjustif, recuo=0cm, abnt-url-package=url, 
abnt-refinfo=yes, abnt-etal-cite=3, abnt-etal-list=3, abnt-thesis-year=final]
{abntex2cite}                  % citações bibliográficas conforme a norma ABNT
%**********************************************%
%				PACOTES				%                      
%**********************************************%
\usepackage[bottom]{footmisc}                               % posicionamento igual das notas de rodapé
\usepackage[utf8]{inputenc}                                 % codificação do documento
\usepackage[T1]{fontenc}                                    % código de fonte                                      
\usepackage{float}                                          % tabelas e figuras no multicol
\usepackage{graphicx}                                       % incluir mídias gráficas
\graphicspath{ {./figuras/} }									% pasta de armazenamento de mídias gráficas
\usepackage{icomma}                                         % vírgulas em equações
\usepackage{subeqnarray}                                    % subnumeração de equações
\usepackage{lastpage}                                       % última página do documento                    
\usepackage{amsfonts, amssymb, amsmath}                     % Fontes e símbolos matemáticos
\usepackage[algoruled, portuguese]{algorithm2e}             % Permite escrever algoritmos em português
\usepackage{ae, aecompl}                                    % Fontes de alta qualidade
\usepackage{latexsym}                                       % Símbolos matemáticos
\usepackage{lscape}                                         % Permite páginas em modo "paisagem"
\usepackage{subfig}                                        % Posicionamento de figuras
\usepackage{times}                                         % Usa a fonte Times
\usepackage{booktabs} 
\usepackage{color}					
\usepackage{indentfirst}                                    
\usepackage{microtype}                                      
\usepackage{multirow, array}                 
\usepackage{hyperref}  
\usepackage{verbatim}
\usepackage{multicol}             
\usepackage{nomencl}[refpage] % lista de siglas %em trabalho
%**********************************************%
%				MACROS				%                      
%**********************************************%
\newcommand{\numero}{N\textsuperscript{\underline{o}}}
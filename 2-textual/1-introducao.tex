%**********************************************%
%			        INTRODUÇÃO			       %                      
%**********************************************%
\chapter{INTRODUÇÃO}
\label{introdução}

A introdução é a primeira seção de um trabalho acadêmico, deve-se explicar de forma
clara a importância da realização do estudo de forma a facilitar a compreensão do leitor.
Nela definem-se brevemente os objetivos, o enfoque dado ao assunto e a relação com
outros estudos. Recomenda-se que seja apresentado em subseções.
Na introdução do trabalho, podem haver citações, Souza e Ilkiu (2017, p. 52) declaram
que:
\begin{citacao}
A introdução é a primeira seção de um trabalho acadêmico, deve-se explicar de
forma clara a importância da realização do estudo de forma a facilitar a compreensão
do leitor. Nela define-se brevemente os objetivos, o enfoque dado ao assunto e a
relação com outros estudos. Recomenda-se que seja apresentado em subseções.
\end{citacao}

%**********************************************%
%			      JUSTIFICATIVA			       %                      
%**********************************************%
\section{JUSTIFICATIVA}
\label{justificativa}

A justificativa em um trabalho de conclusão de curso representa uma apresentação inicial do estudo que pode incluir fatores que motivaram o pesquisador, sua relação e experiência com o tema, argumentação acerca da importância da pesquisa sob o ponto de vista metodológico, teórico ou empírico e a referência ou contribuição para o conhecimento de alguma questão teórica ou prática que ainda não tenha sido solucionada (GIL, 2002).

A justificativa precisa reforçar a importância do tema escolhido e, para isso, alguns estudiosos recomendam incluir citações de autores acerca deste tema para que ocorra um ponto de encontro entre sua ideia e a de outros autores.

Porém, vale ressaltar a importância da narrativa do próprio autor do trabalho de conclusão de curso, sendo que o discurso do autor deve refletir a relevância do tema escolhido e o contexto em que a pesquisa ocorre.
Assim, é preciso evitar na justificativa uma sequência enfadonha de citações a autores exógenos ao trabalho.
%**********************************************%
%			  PROBLEMA DE PESQUISA			   %                      
%**********************************************%
\subsection{Problema de Pesquisa}
\label{problemadepesquisa}
Um problema de pesquisa refere-se a uma pergunta que deve ser respondida pelo
pesquisador, torna-se evidente que é uma questão a ser resolvida por meio da pesquisa
científica. (Não esquecer da pontuação final - ?)
\subsection{Hipóteses}
\label{hipoteses}
A formulação das hipóteses é considerada um dos pontos-chave para a elaboração do
trabalho científico, requer domínio do pesquisador acerca do tema a ser pesquisado e do
problema de pesquisa a ser respondido. Uma hipótese deve ter relação com o problema de
pesquisa.

Alguns aspectos principais na formulação das hipóteses são considerados por Brevidelli
e Sertório (2010, p.43):
\begin{citacao}
Redigir na forma de sentença declarativa, concisa e clara; ser específica e com
referências empíricas; estabelecer uma relação explicativa para o problema de
pesquisa; estabelecer relação quantitativa ou de associação/correlação entre duas ou
mais variáveis.
\end{citacao}

%**********************************************%
%			       OBJETIVOS			       %                      
%**********************************************%
\section{OBJETIVOS}
\label{objetivos}

Após a definição do problema de pesquisa e das hipóteses o pesquisador irá apresentar
os objetivos da pesquisa (objetivo geral e objetivos específicos).
Este fato implica em detalhar quais procedimentos serão realizados para testar suas
hipóteses e responder ao seu problema de pesquisa.

Recomenda-se que a apresentação dos objetivos seja feita a partir da definição de
verbos de ação considerados mensuráveis como: “identificar”, “verificar”, “descrever”,
“mensurar”, “avaliar”, “comparar”, “determinar”, “discutir”, “sintetizar”, dentre outros.
%**********************************************%
%			      OBJETIVO GERAL			   %                      
%**********************************************%
\subsection{Objetivo Geral}
\label{objetivogeral}

É a ação principal do trabalho. O objetivo geral está ligado a uma visão abrangente do
tema. Normalmente é apenas um verbo, exemplo: dimensionar, analisar, projetar, comparar...
%**********************************************%
%			  OBJETIVOS ESPECÍFICOS			   %                      
%**********************************************%
\subsection{Objetivos Específicos}
\label{objetivosespecificos}

Normalmente são 3 verbos. Os objetivos específicos têm característica de apresentar
um caráter mais concreto, pois admitem alcançar o objetivo geral e também a aplicá-lo em
situações específicas, tais como:

\begin{itemize}
    \item analisar...
    \item determinar...  
    \item verificar...
\end{itemize}
%**********************************************%
%			 REFERENCIAL TEÓRICO			   %                      
%**********************************************%
\chapter{REFERENCIAL TEÓRICO}
\label{fundamentacao}

\section{HISTÓRICO}
O referencial teórico em um TCC constitui a parte em que o pesquisador contextualiza o assunto, o problema que se pretende responder com a pesquisa. Refere-se à fundamentação teórica do assunto, às contribuições de outros autores acerca do tema em publicações anteriores.
De acordo com Gil (2002, p. 162) “essa revisão não pode ser constituída apenas por referências ou sínteses dos estudos feitos, mas por discussão crítica do “estado atual da questão”.
O referencial teórico no TCC constitui o item de número 2 na sequência do trabalho e representa a fundamentação teórica da pesquisa. Recomenda-se que os itens sejam apresentados em subdivisões de acordo com o tamanho do texto. As referências são fundamentais para que posteriormente o pesquisador possa realizar a interpretação e a discussão dos resultados.
A busca de evidências, segundo Vitolo (2012, p. 7) “é a técnica que torna possível o encontro entre uma pergunta formulada e a informação armazenada, e as habilidades necessárias para isso são o domínio da ferramenta de busca e a escolha da estratégia adequada”.
\subsection{Tema}

\subsubsection{Descrição}

\subsubsubsection{Quinta seção}
%**********************************************%
%			 MATERIAIS E MÉTODOS			   %                      
%**********************************************%
\chapter{MATERIAIS E MÉTODOS}
\label{materiaisemetodos}

O método cientifico é o item 3 do TCC. É definido como um conjunto de regras que têm por objetivo responder a um problema de pesquisa, ou explicar um fato por meio de hipóteses ou teorias, que podem ser testadas experimentalmente pelo pesquisador, para serem comprovadas ou refutadas (MARCONI, 2001).
O método científico quer descobrir a realidade dos fatos seguindo o caminho da dúvida sistemática, metódica, portanto não se inventa um método. É definido, ainda, como um conjunto de regras seguidas pelo pesquisador afim de produzir novos conceitos e conhecimento. Caracterizado por um conjunto de etapas ou passos a serem seguidos pelo pesquisador.
O método é sistemático, quer descobrir a realidade dos fatos por meio de uma investigação que nasce de um problema observado ou sentido pelo pesquisador. A especificação da metodologia da pesquisa, segundo Marconi (2001, p. 47) “é a que abrange maior número de itens, pois responde, a um só tempo, às questões como? Com quê? Onde? Quanto? Quando? É onde se define onde e como a pesquisa será realizada e ajustada de acordo com as características de cada projeto. Para Ferreira (2001, p. 14): É o local onde se garante a reprodutibilidade da investigação, ou seja, basta ler a metodologia utilizada para qualquer outro investigador reproduzir sua pesquisa em condições absolutamente comparáveis, confirmando ou divergindo dos seus resultados. Para isto, o pesquisador classificará sua pesquisa conforme quadro 15, do Manual da Instituição.

\begin{itemize}
    \item Etapas;
    \item Procedimentos de estudos e coleta de dados;   
    \item Estratégias a serem utilizadas para análise de dados, realização de ensaios, etc.;
		\item Materiais que serão utilizados.
\end{itemize}
%**********************************************%
%			     	  RESUMO			 	   %                      
%**********************************************%
\begin{resumo}[RESUMO]
\begin{SingleSpacing}

\imprimircitacaoautor. \MakeUppercase{\textbf{\imprimirtitulo}}. \textnormal{\imprimirdata. \pageref {LastPage} f. \imprimirprojeto\ – \imprimircurso, \imprimirinstituicao. \imprimirlocal, \imprimirdata}.\\

O resumo é um elemento obrigatório do trabalho (ABNT NBR 6028:2003). Nele deve constar
a natureza do problema de pesquisa, os objetivos do estudo, o método empregado e os
resultados alcançados. Deve ser apresentado em linguagem clara, concisa e objetiva em um
único parágrafo, sem a divisão por itens. Necessita ser redigido na terceira pessoa do tempo
singular e conter entre 150 e 500 palavras. Deve ser composto de uma sequência de frases
concisas, afirmativas e não de enumeração de tópicos. Recomenda-se o uso de parágrafo
único. A primeira frase deve ser significativa, explicando o tema principal. Deve-se usar o
verbo em terceira pessoa.\\

\textbf{Palavras-chave}: Palavra1. Palavra2. Palavra3. (recomenda-se de três a cinco palavras-chave,
antecedidas da expressão Palavras-chave: separadas entre si por ponto e finalizadas também
por ponto.)
\end{SingleSpacing}
\end{resumo}